% Copyright 2017 Marek Smolarczyk

\documentclass[a4paper,12pt]{book}
\usepackage{polski}
\usepackage[utf8]{inputenc}

\usepackage{amssymb, amsmath, calc}
\usepackage{graphicx,graphics}
\usepackage{pgf,tikz}

\usepackage{tasks}
\usepackage{enumitem}

\newcounter{chapc} % Chapter counter
\newcounter{parc} % Paragraph counter

\newcommand{\zadanie}{
	\addtocounter{parc}{1} 
	\paragraph{Zadanie \arabic{chapc}.\arabic{parc}.}
}

% custom chapter without the number in the table of contents
\newcommand{\lightchapter}[2]{
	\setcounter{chapter}{#1}
	\setcounter{section}{0}
	\chapter*{#2}
	\addcontentsline{toc}{chapter}{#2}
}

\begin{document}

\frontmatter

\begin{titlepage}
	\centering
	{ \bfseries \LARGE Matematyka - poziom rozszerzony \par }
	\vspace{1cm}
	{ \large Zbiór zadań przygotowujący do matury z matematyki na poziomie rozszerzonym \par }
	\vspace{3cm}
	{ \itshape \large Marek Smolarczyk \par }
	\vfill
	{ \large v. 1.1.1. \par }
	\vspace{0.2cm}
	{ \large \today \par }
\end{titlepage}

\author{Marek Smolarczyk}
\title{Matematyka - poziom rozszerzony}

\tableofcontents

\mainmatter
\setcounter{parc}{0}
\addtocounter{chapc}{1}

\chapter{Funkcje}

\section{Zadania otwarte}

\zadanie Czy istnieje funkcja jednocześnie parzysta i nieparzysta? Jeśli tak, to podaj przykład, a jeśli nie, to udowodnij dlaczego taka funkcja nie istnieje.
%\begin{sol}
%	Tak, funkcja stale równa $0$.
%\end{sol}

\zadanie Czy istnieje funkcja okresowa, której okresem jest każda liczba rzeczywista $t > 0$? Jeśli tak, to podaj przykład, a jeśli nie, to udowodnij dlaczego taka funkcja nie istnieje.
%\begin{sol}
%	Tak, dowolna funkcja stała.
%\end{sol}

\zadanie Wyznacz dziedzinę funkcji log$_{x^2 - 4x + 1}(x^2 + 3x)$.

\zadanie Wyznacz zbiór wartości funkcji $2\cdot|5x^2 - 2|$ określonej na przedziale zamkniętym $\langle -1, 1 \rangle$.

\zadanie Wykonaj wykres funkcji $f(x) = |log_2(3x + 6)|$ i wyznacz wszystkie wartości parametru $k$, dla których równanie $f(x) = k - 1$ ma dwa pierwiastki przeciwnych znaków.

\zadanie Wykonaj wykres funkcji $f(x) = |3^{|x - 1|} - 2|$ i wyznacz ilość rozwiązań równania $f(x) = k$ w zależności od parametru $k$.

\zadanie Niech $f(x) = |x^2 -3|x| + 2|$. Wyznacz ilość rozwiązań równania $f(x) = k$ w zależności od parametru $k$.

\zadanie Udowodnij, że funkcja $f(x) = x^2 + $log$\,x$ określona dla $x \in \mathbb{R}_+$ jest różnowartościowa.

 % Funkcje
\chapter{Wielomiany}

\section{Zadania zamknięte}

\paragraph{Zadanie 1.1.} Reszta z dzielenia wielomianu $x^3 - 4x^2 + x -3$ przez $2x - 1$ wynosi
\begin{tasks}(4)
	\task $-\dfrac{5}{2}$
	\task $-5$
	\task $-9$
	\task $-\dfrac{27}{8}$ %correct%
\end{tasks}

\paragraph{Zadanie 1.2.} Ile różnych rozwiązań w zależności od parametru $m$ może mieć równanie $x^3 - 2mx^2 - 6x + 12m = 0$?
\begin{enumerate}[label=\alph*)]
	\item Jedno lub dwa rozwiązania
	\item Tylko dwa rozwiązania
	\item Dwa lub trzy rozwiąznia %correct%
	\item Tylko trzy rozwiązania
\end{enumerate}

\paragraph{Zadanie 1.3.} Liczba $3$ jest pierwiastkiem wielomianu ${x^4 - 4x^3 + 6x^2 - (k + 3)x + 9}$. Wartość $k$ wynosi
\begin{tasks}(4)
	\task $9$ %correct%
	\task $-15$
	\task $3$
	\task $-6$
\end{tasks}

\newpage

\paragraph{Zadanie 1.4.} Wielomiany $W(x) = x^4 + 2x^3 - 13x^2 - 38x - 24$ oraz\\ ${Q(x) = (a - b)x^4 + 2x^3 +(2b - 1)x^2 -38x +6a - b}$ są równe dla wartości parametrów
\begin{enumerate}[label=\alph*)]
	\item takie wartości nie istnieją
	\item $a = -3$, $b = -6$
	\item $a - -3$, $b = -4$
	\item $a = -5$, $b = -6$ %correct%
\end{enumerate}

\paragraph{Zadanie 1.5.} Resztą z dzielenia wielomianu $-x^3 - 3x^2 + kx +8$ przez dwumian $x - 1$ jest równa $10$. Wartość parametru $k$ wynosi
\begin{tasks}(4)
	\task $4$
	\task $-4$
	\task $6$ %correct%
	\task $-6$
\end{tasks}

\paragraph{Zadanie 1.6.} Wielomian stopnia trzeciego $W(x)$ jest funkcją nieparzystą i $W(1) = 0$. Ile różnych pierwiastków może mieć ten wielomian?
\begin{enumerate}[label=\alph*)]
	\item Tylko jeden pierwiastek
	\item Jeden lub trzy pierwiastki
	\item Tylko trzy pierwiastki
	\item Taki wielomian nie istnieje
\end{enumerate}

\section{Zadania otwarte}

\paragraph{Zadanie 1.?.} Wielomian $x^3 + bx^2 + cx + 4$ ma trzy pierwiastki rzeczywiste równe $x_1,\, x_2,\, x_3$. Wiedząc, że reszta z dzielenia tego wielomianu przez trójmian $x^2 + 2$ wynosi $-6x + 8$, wyznacz wartość wyrażenia $x_1(x_2 + x_3 + 1) + x_2(x_3 + 1) + x_3$. %-2%

\paragraph{Zadanie 1.?.} Wyznacz resztę z dzielenia wielomianu $W(x)$ przez trójmian $(x - 5)(x + 1)$ wiedząc, że $W(5) = 10$ i $W(-1) = 4$. %x + 5%

\paragraph{Zadanie 1.?.} Pan Jan posiada $60m$ płotu i chce ogrodzić swoją posesję, tak żeby wybudowane ogrodzenie było w kształcie prostokąta. Na południowej stronie działki stoi już brama o długości dwóch metrów, która ma stanowić wejście na ogrodzoną posesję. Wyznacz funckję $W(x)$, która dla długości wschodniej części płotu w metrach, zwróci pole powierzchni ogrodzonej działki (w $m^2$). Policz, dla płotu o jakich wymiarach, pole powierzchni ogrodzonej działki będzie największe. %W(x) = x(31 - x)  y_max = W(15.5)% % Wielomiany
\setcounter{parc}{0}
\addtocounter{chapc}{1}

\chapter{Planimetria}

\section{Zadania zamknięte}

\section{Zadania otwarte}

\zadanie Jak znaleźć odległość między dwoma punktami, do których nie możemy dojść? Załóżmy, że chcemy zmierzyć odległość pomiędzy dwoma drzewami, które znajdują się po przeciwnej stronie rzeki. Jak musimy postępować, aby zmierzyć odległość między tymi drzewami, bez przechodzenia przez rzekę? Zakładamy, że potrafimy zmierzyć odległość pomiędzy dwoma dowolnymi punktami po naszej stronie rzeki. % Planimetria
\chapter{Rachunek różniczkowy}

\section{Zadania zamknięte}

%TODO: Odpowiedzi nie są jednoznaczne! %
\paragraph{Zadanie 2.1.} Funkcja $f(x) = 3x^3 + x - 3$ jest
\begin{enumerate}[label=\alph*)]
	\item rosnąca w przedziale $ \langle -\sqrt{3}; \sqrt{3} \rangle $
	\item rosnąca w przedziale $ ( -\infty; -\sqrt{3} \rangle \cup \langle \sqrt{3}; +\infty ) $
	\item rosnąca w przedziale $ ( -\infty; -\frac{1}{3} \rangle \cup \langle \frac{1}{3}; +\infty ) $
	\item rosnąca w całej swojej dziedzinie
\end{enumerate}

\paragraph{Zadanie 2.2.}  % Rachunek różniczkowy


\lightchapter{\value{chapc}}{Odpowiedzi i rozwiązania}

\odpowiedzi
\begin{Soln}{d)}
			d)
		
\end{Soln}


 % odpowiedzi i rozwiązania zadań wszystkich rozdziałów

\backmatter
% bibliography, glossary and index would go here.

\end{document}
