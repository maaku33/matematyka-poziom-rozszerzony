\setcounter{parc}{0}
\addtocounter{chapc}{1}

\chapter{Wielomiany}

\section{Zadania zamknięte}
\begin{zadania}
	\begin{zad} 
		Reszta z dzielenia wielomianu $x^3 - 4x^2 + x -3$ przez $2x - 1$ wynosi
		\begin{tasks}(4)
			\task $-\dfrac{5}{2}$
			\task $-5$
			\task $-9$
			\task $-\dfrac{27}{8}$ %correct
		\end{tasks}
		\begin{sol}
			d)
		\end{sol}
	\end{zad}
\end{zadania}

\zadanie Ile różnych rozwiązań w zależności od parametru $m$ może mieć równanie $x^3 - 2mx^2 - 6x + 12m = 0$?
\begin{enumerate}[label=\alph*)]
	\item Jedno lub dwa rozwiązania
	\item Tylko dwa rozwiązania
	\item Dwa lub trzy rozwiązania %correct
	\item Tylko trzy rozwiązania
\end{enumerate}

\zadanie Liczba $3$ jest pierwiastkiem wielomianu ${x^4 - 4x^3 + 6x^2 - (k + 3)x + 9}$. Wartość $k$ wynosi
\begin{tasks}(4)
	\task $9$ %correct
	\task $-15$
	\task $3$
	\task $-6$
\end{tasks}

\newpage

\zadanie Wielomiany $W(x) = x^4 + 2x^3 - 13x^2 - 38x - 24$ oraz\\ ${Q(x) = (a - b)x^4 + 2x^3 +(2b - 1)x^2 -38x +6a - b}$ są równe dla wartości parametrów
\begin{enumerate}[label=\alph*)]
	\item takie wartości nie istnieją
	\item $a = -3$, $b = -6$
	\item $a - -3$, $b = -4$
	\item $a = -5$, $b = -6$ %correct
\end{enumerate}

\zadanie Resztą z dzielenia wielomianu $-x^3 - 3x^2 + kx +8$ przez dwumian $x - 1$ jest równa $10$. Wartość parametru $k$ wynosi
\begin{tasks}(4)
	\task $4$
	\task $-4$
	\task $6$ %correct
	\task $-6$
\end{tasks}

\zadanie Wielomian stopnia trzeciego $W(x)$ jest funkcją nieparzystą i $W(1) = 0$. Ile różnych pierwiastków może mieć ten wielomian?
\begin{enumerate}[label=\alph*)]
	\item Tylko jeden pierwiastek
	\item Jeden lub trzy pierwiastki
	\item Tylko trzy pierwiastki %correct
	\item Taki wielomian nie istnieje
\end{enumerate}

\zadanie Dany jest wielomian $W(x) = x^4 - 7x^2 + 6x$. Zaznacz prawdziwe stwierdzenie.
\begin{enumerate}[label=\alph*)]
	\item Równanie $W(x) = 0$ ma dwa rozwiązania dodatnie. %correct
	\item Równanie $W(x) = 0$ ma dwa rozwiązania ujemne.
	\item Równanie $W(x) = 0$ ma tylko dwa rozwiązania.
	\item $W(x)$ nie da się zapisać w postaci iloczynu czterech czynników stopnia pierwszego.
\end{enumerate}

\zadanie Wiadomo, że wielomian $W(x) = x^3 + ax^2 - ax - 1$ jest podzielny przez dwumian $x + 1$. Zatem reszta z dzielenia tego wielomianu przez dwumian $x + 2$ jest równa
\begin{tasks}(4)
	\task $x + 1$
	\task $-3$ %correct
	\task $1$
	\task $7$
\end{tasks}

\section{Zadania otwarte}

\zadanie Wielomian $x^3 + bx^2 + cx + 4$ ma trzy pierwiastki rzeczywiste równe $x_1,\, x_2,\, x_3$. Wiedząc, że reszta z dzielenia tego wielomianu przez trójmian $x^2 + 2$ wynosi $-6x + 8$, wyznacz wartość wyrażenia $x_1(x_2 + x_3 + 1) + x_2(x_3 + 1) + x_3$. %-2

\zadanie Wyznacz resztę z dzielenia wielomianu $W(x)$ przez trójmian $(x - 5)(x + 1)$ wiedząc, że $W(5) = 10$ i $W(-1) = 4$. %x + 5

\zadanie Pan Jan posiada $60m$ płotu i chce ogrodzić swoją posesję, tak żeby wybudowane ogrodzenie było w kształcie prostokąta. Na południowej stronie działki stoi już brama o długości dwóch metrów, która ma stanowić wejście na ogrodzoną posesję. Wyznacz funkcję $f$, która dla długości wschodniej części płotu w metrach, zwróci pole powierzchni ogrodzonej działki (w~$m^2$). Podaj wymiary płotu, dla którego pole powierzchni ogrodzonej działki będzie największe. %W(x) = x(31 - x)  y_max = W(15.5)

\zadanie Pierwiastki wielomianu $2x^3 + x^2 - 13x + 6$ są liczbami wymiernymi. Wyznacz te pierwiastki i zapisz dany wielomian w postaci iloczynu czynników maksymalnie pierwszego stopnia. %2(x+3)(x-2)(x - 1/2)  twierdzenie o pierwiastkach wymiernych

\zadanie Liczby $a, b, c$ są pierwiastkami wielomianu $W(x) = x^3 - 2x^2 - 23x + 60$. Oblicz wartość wyrażenia $(a + 1)(b + 1)(c + 1)$. % -80

\zadanie Wyznacz wszystkie wartości parametru $m$, dla których pierwiastkiem trójmianu $8x^2+(2m+1)x+2m-1$ jest właśnie wartość $m$. %-1/2 1/5

\zadanie Wyznacz wszystkie wartości parametru $m$, dla których suma odwrotności dwóch różnych pierwiastków równania \[ {(m+2)x^2+2mx+1=0} \] jest mniejsza od $8$.

\zadanie Wyznacz wszystkie wartości parametru $m$, dla których mniejszy z pierwiastków równania \[ (m-2)x^2+(2m+1)x+1=0 \] jest mniejszy niż $1$, ale większy od $-2$.

\zadanie Znajdź wielomian $W(x)$, który jest podzielny przez trójmian $x^2 - 1$, jego pierwiastkiem jest liczba $3$, a jego wykres przecina oś rzędnych w~punkcie $(0, 6)$. %2(x-1)(x+1)(x-3)