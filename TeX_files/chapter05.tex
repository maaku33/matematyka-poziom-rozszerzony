\setcounter{parc}{0}
\addtocounter{chapc}{1}

\chapter{Trygonometria}

\section{Zadania zamknięte}

\zadanie Jeżeli $\alpha, \beta, \gamma$ są miarami kątów wewnętrznych trójkąta, to zachodzi
\begin{enumerate}[label=\alph*)]
	\item sin$(\alpha + \beta)$ = sin\,$\gamma$ %correct
	\item cos$(\alpha + \beta)$ = cos\,$\gamma$
	\item sin$(\alpha + \beta)$ = $-$sin\,$\gamma$
	\item cos$(\alpha + \beta)$ = $-$cos\,$\gamma$
\end{enumerate}
%\begin{sol}
%	a)
%\end{sol}

\section{Zadania otwarte}

\zadanie Znajdź wszystkie rozwiązania równania $\text{cos}(2x) + 2\text{sin}(x) = \frac{3}{2}$ w przedziale $\langle 0; 2\pi \rangle$.

\zadanie Rozwiąż równanie $-\text{cos}^3\,x -3\text{sin}^2\,x + 6 = 5\text{cos}\,x$ dla liczb rzeczywistych.

\zadanie Znajdź wszystkie rozwiązania równania $\text{sin}\, x|\text{cos}\,x| = 0,25$ w przedziale $\langle 0; 2\pi \rangle$.

\zadanie Dane jest równanie $\text{sin}\,x\cdot\text{cos}\,x = a^2 - 1$. Wyznacz wszystkie wartości parametru $a$, dla których dane równanie nie ma rozwiązań.

\zadanie Znajdź wszystkie rozwiązania równania \[ \sin(15\degree+2x) + \sin(15\degree-2x) = \sqrt{3}\sin15\degree \]

\zadanie Dany jest trójkąt prostokątny, taki że suma cosinusów jego kątów ostrych równa się $\frac{2\sqrt{3}}{3}$. Policz iloczyn sinusów kątów ostrych tego trójkąta.

\zadanie Rozwiąż równanie $ \frac{\sqrt{3}}{3}\text{sin}\,2x = -\text{cos}\,x $. 

\zadanie Rozwiąż nierówność \[ \dfrac{2\text{sin}\, x - 1}{\text{sin}^2\, x} < 0 \]

\zadanie Rozwiąż nierówność $ \text{tg}\,x + \text{ctg}\,x < 2\text{sin}\,2x$.

\zadanie Oblicz $ 3\,\text{ctg}\,\alpha $ wiedząc, że $\alpha$ jest kątem ostrym i $\text{sin}\,\alpha = \frac{3}{5} $.

\zadanie Kąt $\alpha$ jest ostry oraz \[ \dfrac{16}{\text{sin}^2\, \alpha} + \dfrac{16}{\text{cos}^2 \alpha} = 9. \] Policz iloczyn $\text{sin}\, \alpha\cdot\text{cos}\, \alpha$.

\zadanie Znajdź wartość wyrażenia \[\text{tg}^2\,\alpha + \bigg(\frac{1}{\text{tg}\,\alpha}\bigg)^2 \] wiedząc, że $\text{tg}\,\alpha + \frac{1}{\text{tg}\,\alpha} = 4$ i $\alpha$ jest kątem ostrym.

\zadanie Udowodnij tożsamość \[ \tan^2(x) - \sin^2(x) = \tan^2(x)\sin^2(x) \]