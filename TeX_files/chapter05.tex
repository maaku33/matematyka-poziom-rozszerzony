\setcounter{parc}{0}
\addtocounter{chapc}{1}

\chapter{Trygonometria}

\section{Zadania zamknięte}

\zadanie Jeżeli $\alpha, \beta, \gamma$ są miarami kątów wewnętrznych trójkąta, to zachodzi
\begin{enumerate}[label=\alph*)]
	\item sin$(\alpha + \beta)$ = sin\,$\gamma$ %correct
	\item cos$(\alpha + \beta)$ = cos\,$\gamma$
	\item sin$(\alpha + \beta)$ = $-$sin\,$\gamma$
	\item cos$(\alpha + \beta)$ = $-$cos\,$\gamma$
\end{enumerate}
%\begin{sol}
%	a)
%\end{sol}

\section{Zadania otwarte}

\zadanie Rozwiąż równanie $-\text{cos}^3\,x -3\text{sin}^2\,x + 6 = 5\text{cos}\,x$ dla liczb rzeczywistych.

\zadanie Znajdź wszystkie rozwiązania równania $\text{sin}\, x|\text{cos}\,x| = 0,25$ w przedziale $\langle 0; 2\pi \rangle$.

\zadanie Rozwiąż nierówność \[ \dfrac{2\text{sin}\, x - 1}{\text{sin}^2\, x} < 0 \]

\zadanie Kąt $\alpha$ jest ostry oraz \[ \dfrac{16}{\text{sin}^2\, \alpha} + \dfrac{16}{\text{cos}^2 \alpha} = 9. \] Policz iloczyn $\text{sin}\, \alpha\cdot\text{cos}\, \alpha$.

\zadanie Znajdź wartość wyrażenia \[\text{tg}^2\,\alpha + \bigg(\frac{1}{\text{tg}\,\alpha}\bigg)^2 \] wiedząc, że $\text{tg}\,\alpha + \frac{1}{\text{tg}\,\alpha} = 4$ i $\alpha$ jest kątem ostrym.