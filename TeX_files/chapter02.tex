\setcounter{parc}{0}
\addtocounter{chapc}{1}

\chapter{Rachunek różniczkowy}

\section{Zadania zamknięte}

\zadanie Funkcja $f(x) = 3x^3 + x - 3$ jest
\begin{enumerate}[label=\alph*)]
	\item rosnąca w przedziale $ \langle -\sqrt{3}; \sqrt{3} \rangle $ i malejąca w $ ( -\infty; -\sqrt{3} \rangle \cup \langle \sqrt{3}; +\infty ) $
	\item rosnąca w przedziale $ ( -\infty; -\sqrt{3} \rangle \cup \langle \sqrt{3}; +\infty ) $ i malejąca w $ \langle -\sqrt{3}; \sqrt{3} \rangle $
	\item rosnąca w przedziale $ ( -\infty; -\frac{1}{3} \rangle \cup \langle \frac{1}{3}; +\infty ) $ i malejąca w $ \langle -\frac{1}{3}; \frac{1}{3} \rangle $
	\item rosnąca w całej swojej dziedzinie
\end{enumerate}

\zadanie Równanie $($sin$^22x)^{'}=0$ ma w przedziale $\langle 0, \pi \rangle$
\begin{enumerate}[label=\alph*)]
	\item dokładnie dwa rozwiązania
	\item dokładnie cztery rozwiązania
	\item dokładnie pięć rozwiązań % correct
	\item nieskończenie wiele rozwiązań
\end{enumerate}

\section{Zadania otwarte}

\zadanie Znajdź wszystkie wartości parametru $a\in\mathbb{R}$, dla których funkcja $f(x) = x^5 - 10x^2 + ax$ jest różnowartościowa.

\zadanie Udowodnij, że funkcja $f(x) = x^2 - x\,+\,$log$\,x$ określona dla $x \in \mathbb{R}_+$ jest różnowartościowa.

\zadanie Dane są dwie parabole o równaniach $y = x^2 + 1$ i $y =-x^2 - 2$. Wyznacz równania wszystkich takich prostych, że są one jednocześnie styczne do wykresów obu parabol.

\zadanie Wykaż, że wielomian $ \frac{1}{2}x^4 - 2x^3 + 8x + 6 $ nie posiada pierwiastków rzeczywistych.