\setcounter{parc}{0}
\addtocounter{chapc}{1}

\chapter{Planimetria}

\section{Zadania zamknięte}

\section{Zadania otwarte}

\zadanie Jak znaleźć odległość między dwoma punktami, do których nie możemy dojść? Załóżmy, że stoimy przy rzece, a po jej przeciwnej stronie stoją dwa drzewa. Jak musimy postępować, aby zmierzyć odległość między tymi drzewami, bez przechodzenia przez rzekę? Zakładamy, że potrafimy zmierzyć odległość pomiędzy dwoma dowolnymi punktami po naszej stronie rzeki, oraz wyznaczać dokładne kąty proste.

\zadanie Mamy dane dwa współśrodkowe okręgi o promieniach długości $3$ i $1$. Cięciwa $AB$ większego okręgu jest styczna do mniejszego okręgu. Ile wynosi pole koła o średnicy $AB$?

\zadanie W trójkąt $ABC$ o bokach $|AC| = 13, |BC| = 15, |AB| = 14$ wpisano półokrąg, tak, że jego środek leży na boku $AB$ i jest on styczny do boków $AC$ i $BC$. Oblicz długość promienia tego półokręgu.