\setcounter{parc}{0}
\addtocounter{chapc}{1}

\chapter{Ciągi liczbowe}

\section{Zadania zamknięte}

\section{Zadania otwarte}

\zadanie Dany jest ciąg $(a_n)$, gdzie $a_n = n(n+1)$, dla $n > 0$. Udowodnij, że suma pierwszych $n$ wyrazów tego ciągu wyraża się wzorem \[S_n = \dfrac{n(n+1)(n+2)}{3}\]

\zadanie Ciąg $(a_n)$ dany jest wzorem rekurencyjnym \[a_{n+3} = a_{n+2} + a_{n+1} - a_{n}\,\text{,  dla } n > 0\] Wyznacz wartość $a_6$ jeśli $a_1 = 1$, $a_2 = 2$ i $a_3 = 4$.

\zadanie Ciąg $(a_n)$ dany jest wzorem rekurencyjnym \[a_{n+3} = a_{n+2} - a_{n+1} + a_{n}\,\text{,  dla } n \geqslant 0\] Wyznacz wartość $a_{999}$ jeśli $a_0 = 2$, $a_1 = 1$, $a_2 = 2$. \\{\footnotesize\textit{Wskazówka}: wypisz kilka początkowych wyrazów ciągu.}

\zadanie Suma pierwszych $n$ wyrazów ciągu $(a_n)$ wyraża się wzorem \\${\frac{3}{2}n^2 + \frac{7}{2}n}$. Pokaż, że jest to ciąg arytmetyczny i wyznacz wartość jedenastego wyrazu ciągu.

\zadanie Dany jest ciąg arytmetyczny $(a_n)$, gdzie $a_{10} = 20$ i $a_{50} = -10$. Wyznacz $a_{200}$.

\zadanie Ciąg $(2x,\;3x - 1,\;6x - 4)$ jest ciągiem arytmetycznym. Znajdź wartość $x$. Wyznacz różnicę tego ciągu.

\zadanie Wykaż, że suma $n$ kolejnych liczb nieparzystych, zaczynając od $1$, jest kwadratem liczby naturalnej.

\zadanie Ciąg arytmetyczny dany jest wzorem $a_n = -1 + 4(n - 1)$, dla $n > 0$. Znajdź sumę wszystkich wyrazów $a_i$, takich że $i < 100$ oraz $i$ jest podzielne przez $2$ lub $5$.

\zadanie Niech $S_n = 2^n - 1$. Pokaż, że ciąg $(S_n)$ jest ciągiem sum częściowych szeregu geometrycznego.

\zadanie Ciąg sum częściowych ciągu $(c_n)$ jest ciągiem arytmetycznym o pierwszym wyrazie równym $s_1 = -1$. Wyznacz ciąg $(c_n)$.

\zadanie Dany jest ciąg $(g_n)$, taki że $g_1 = g_2 = 1$ i $(g_n)$ dla $n > 1$ jest ciągiem geometrycznym o ilorazie $2$. Pokaż, że jego ciąg sum częściowych $(S_n)$ dla $n > 0$ jest również ciągiem geometrycznym.

\zadanie Ciąg $(5x - 12,\;x,\;8 - x)$ jest ciągiem geometrycznym. Znajdź wszystkie możliwe ilorazy tego ciągu. % 2 i 1/3

\zadanie Ciąg $(a, b, c)$ jest ciągiem arytmetycznym i suma jego wyrazów wynosi $33$. Wyznacz ten ciąg wiedząc, że $(a - 5, b - 1, c + 8)$ jest ciągiem geometrycznym. 

\zadanie Ciąg dany jest wzorem $c_n = \frac{16}{2^n}$, dla $n > 0$. Wyznacz sumę wszystkich $c_i$, takich że $2|i$ lub $3|i$. % 464/63 ??

\zadanie Kolejne kąty wewnętrzne pięciokąta tworzą ciąg arytmetyczny o różnicy $10^{\circ}$. Wyznacz miarę największego kąta. % 128 degrees